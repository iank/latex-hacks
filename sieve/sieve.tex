% $ time pdflatex sieve.tex
% real    0m1.412s
% user    0m1.346s
% sys     0m0.050s


\documentclass{article}
\errorcontextlines 10000
\usepackage{ifthen}
\usepackage{forloop}
\usepackage{intcalc}
\usepackage{etoolbox}
\usepackage{amsmath}
\usepackage{xcolor}
\usepackage{url}


% http://tex.stackexchange.com/questions/37426/create-an-array-of-variables
% http://tex.stackexchange.com/questions/60585/how-to-highlight-a-single-element-in-a-matrix

\newcommand\settext[2]{%
  \csdef{text#1}{#2}}
\newcommand\gettext[1]{%
  \csuse{text#1}}


\author{Ian Kilgore}
\date{2014-09-11}
\title{Programmatically-generated \LaTeX\:Sieve of Eratosthenes}
\begin{document}

\maketitle

\section{Background}

See \url{https://github.com/iank/latex-hacks/blob/master/sieve/sieve.tex} for source code. This \LaTeX code implements a \url{http://en.wikipedia.org/wiki/Sieve_of_Eratosthenes} using:

\begin{itemize}
    \item \verb|\forloop| (package: forloop)
    \item \verb|\ifthenelse| (package: ifthen)
    \item package intcalc
    \item An ``array'' constructed by using indices in macro names \url{http://tex.stackexchange.com/questions/37426/create-an-array-of-variables}.
\end{itemize}



\newcounter{i}
\newcounter{q}
\newcounter{j}
\def \MAX {100}
\def \WRAP {10}

% Initialize sieve (0 == composite, 1 == prime/not marked)
\forloop{i}{2}{\thei < \MAX}{
    \settext{\thei}{1}
}



\newcommand{\hlightP}[1]{%
  \ooalign{\hss\makebox[0pt]{\fcolorbox{red!30}{green!40}{$#1$}}\hss\cr\phantom{$#1$}}%
}
\newcommand{\hlightC}[1]{%
  \ooalign{\hss\makebox[0pt]{\fcolorbox{green!30}{red!40}{$#1$}}\hss\cr\phantom{$#1$}}%
}
\def\tand{&}

\newcommand\printSieve{
\begin{equation*}
    \begin{pmatrix}
        \tand \forloop{q}{2}{\theq < \MAX}{
            \ifthenelse{\equal{\gettext{\theq}}{0}}{\hlightC{\arabic{q}}}{\hlightP{\arabic{q}}}
            %\ifthenelse{\equal{\gettext{\theq}}{0}}{\arabic{q}}{\hlight{\arabic{q}}}
            \ifthenelse{\equal{\intcalcMod{\theq}{\WRAP}}{0}}{ \\ } { \tand }
        }

    \end{pmatrix}
\end{equation*}
}

\section{Blank sieve}
\printSieve

% loop over all primes p and mark their multiples
\forloop{i}{2}{\thei < \MAX}{
    \ifthenelse{ \equal{\gettext{\thei}}{1} }{
        \section{Marking multiples of p = \arabic{i}}
        % mark all multiples of p
        \forloop{j}{2}{\intcalcMul{\thej}{\thei} < \MAX}{
            \settext{\intcalcMul{\thej}{\thei}}{0}
        }
        \printSieve
    } { 
        % else continue
    }
}

\end{document}
